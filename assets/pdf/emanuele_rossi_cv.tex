\documentclass[10pt, letterpaper]{article}

% Packages:
\usepackage[
    ignoreheadfoot, % set margins without considering header and footer
    top=2 cm, % seperation between body and page edge from the top
    bottom=2 cm, % seperation between body and page edge from the bottom
    left=2 cm, % seperation between body and page edge from the left
    right=2 cm, % seperation between body and page edge from the right
    footskip=1.0 cm, % seperation between body and footer
    % showframe % for debugging 
]{geometry} % for adjusting page geometry
\usepackage{titlesec} % for customizing section titles
\usepackage{tabularx} % for making tables with fixed width columns
\usepackage{array} % tabularx requires this
\usepackage[dvipsnames]{xcolor} % for coloring text
\definecolor{primaryColor}{RGB}{0, 79, 144} % define primary color
\usepackage{enumitem} % for customizing lists
\usepackage{fontawesome5} % for using icons
\usepackage{amsmath} % for math
\usepackage[
    pdftitle={Emanuele Rossi's CV},
    pdfauthor={Emanuele Rossi},
    pdfcreator={LaTeX with RenderCV},
    colorlinks=true,
    urlcolor=primaryColor
]{hyperref} % for links, metadata and bookmarks
\usepackage[pscoord]{eso-pic} % for floating text on the page
\usepackage{calc} % for calculating lengths
\usepackage{bookmark} % for bookmarks
\usepackage{lastpage} % for getting the total number of pages
\usepackage{changepage} % for one column entries (adjustwidth environment)
\usepackage{paracol} % for two and three column entries
\usepackage{ifthen} % for conditional statements
\usepackage{needspace} % for avoiding page brake right after the section title
\usepackage{iftex} % check if engine is pdflatex, xetex or luatex

% Ensure that generate pdf is machine readable/ATS parsable:
\ifPDFTeX
    \input{glyphtounicode}
    \pdfgentounicode=1
    % \usepackage[T1]{fontenc} % this breaks sb2nov
    \usepackage[utf8]{inputenc}
    \usepackage{lmodern}
\fi



% Some settings:
\AtBeginEnvironment{adjustwidth}{\partopsep0pt} % remove space before adjustwidth environment
\pagestyle{empty} % no header or footer
\setcounter{secnumdepth}{0} % no section numbering
\setlength{\parindent}{0pt} % no indentation
\setlength{\topskip}{0pt} % no top skip
\setlength{\columnsep}{0cm} % set column seperation
\makeatletter
\let\ps@customFooterStyle\ps@plain % Copy the plain style to customFooterStyle
\patchcmd{\ps@customFooterStyle}{\thepage}{
    \color{gray}\textit{\small Emanuele Rossi - Page \thepage{} of \pageref*{LastPage}}
}{}{} % replace number by desired string
\makeatother
\pagestyle{customFooterStyle}

\titleformat{\section}{\needspace{4\baselineskip}\bfseries\large}{}{0pt}{}[\vspace{1pt}\titlerule]

\titlespacing{\section}{
    % left space:
    -1pt
}{
    % top space:
    0.3 cm
}{
    % bottom space:
    0.2 cm
} % section title spacing

\renewcommand\labelitemi{$\circ$} % custom bullet points
\newenvironment{highlights}{
    \begin{itemize}[
        topsep=0.10 cm,
        parsep=0.10 cm,
        partopsep=0pt,
        itemsep=0pt,
        leftmargin=0.4 cm + 10pt
    ]
}{
    \end{itemize}
} % new environment for highlights

\newenvironment{highlightsforbulletentries}{
    \begin{itemize}[
        topsep=0.10 cm,
        parsep=0.10 cm,
        partopsep=0pt,
        itemsep=0pt,
        leftmargin=10pt
    ]
}{
    \end{itemize}
} % new environment for highlights for bullet entries


\newenvironment{onecolentry}{
    \begin{adjustwidth}{
        0.2 cm + 0.00001 cm
    }{
        0.2 cm + 0.00001 cm
    }
}{
    \end{adjustwidth}
} % new environment for one column entries

\newenvironment{twocolentry}[2][]{
    \onecolentry
    \def\secondColumn{#2}
    \setcolumnwidth{\fill, 4.5 cm}
    \begin{paracol}{2}
}{
    \switchcolumn \raggedleft \secondColumn
    \end{paracol}
    \endonecolentry
} % new environment for two column entries

\newenvironment{header}{
    \setlength{\topsep}{0pt}\par\kern\topsep\centering\linespread{1.5}
}{
    \par\kern\topsep
} % new environment for the header

\newcommand{\placelastupdatedtext}{% \placetextbox{<horizontal pos>}{<vertical pos>}{<stuff>}
  \AddToShipoutPictureFG*{% Add <stuff> to current page foreground
    \put(
        \LenToUnit{\paperwidth-2 cm-0.2 cm+0.05cm},
        \LenToUnit{\paperheight-1.0 cm}
    ){\vtop{{\null}\makebox[0pt][c]{
        \small\color{gray}\textit{Last updated in April 2025}\hspace{\widthof{Last updated in April 2025}}
    }}}%
  }%
}%

% save the original href command in a new command:
\let\hrefWithoutArrow\href

% new command for external links:
\renewcommand{\href}[2]{\hrefWithoutArrow{#1}{\ifthenelse{\equal{#2}{}}{ }{#2 }\raisebox{.15ex}{\footnotesize \faExternalLink*}}}


\begin{document}
    \newcommand{\AND}{\unskip
        \cleaders\copy\ANDbox\hskip\wd\ANDbox
        \ignorespaces
    }
    \newsavebox\ANDbox
    \sbox\ANDbox{}

    \begin{header}
        \textbf{\fontsize{24 pt}{24 pt}\selectfont Emanuele Rossi}

        \vspace{0.3 cm}

        \normalsize
        \mbox{{\color{black}\footnotesize\faMapMarker*}\hspace*{0.13cm}Barcelona}%
        \kern 0.25 cm%
        \AND%
        \kern 0.25 cm%
        \mbox{\hrefWithoutArrow{mailto:emanuele.rossi1909@gmail.com}{\color{black}{\footnotesize\faEnvelope[regular]}\hspace*{0.13cm}emanuele.rossi1909@gmail.com}}%
        \kern 0.25 cm%
        \AND%
        \kern 0.25 cm%
        \mbox{\hrefWithoutArrow{https://www.emanuelerossi.co.uk/}{\color{black}{\footnotesize\faLink}\hspace*{0.13cm}emanuelerossi.co.uk}}%
        \kern 0.25 cm%
        \AND%
        \kern 0.25 cm%
        \mbox{\hrefWithoutArrow{https://linkedin.com/in/emanuele-rossi/}{\color{black}{\footnotesize\faLinkedinIn}\hspace*{0.13cm}emanuele-rossi}}%
        \kern 0.25 cm%
        \AND%
        \kern 0.25 cm%
        \mbox{\hrefWithoutArrow{https://github.com/emalgorithm}{\color{black}{\footnotesize\faGithub}\hspace*{0.13cm}emalgorithm}}%
    \end{header}

    \vspace{0.3 cm - 0.3 cm}

    % \section{Summary}
    
    % \begin{onecolentry}
    %     I'm an ML researcher specializing in generative models for structural biology. Before jumping into MLxBio, my background combined \textbf{fundamental research} experience with real-world \textbf{product impact}. I'm the first author of \href{https://arxiv.org/abs/2004.11198}{Scalable Inception Graph Network}, now \href{https://medium.com/airbnb-engineering/graph-machine-learning-at-airbnb-f868d65f36ee}{\textbf{used in production by AirBnB}}, and \href{https://arxiv.org/abs/2006.10637}{Temporal Graph Network}, which has received over \href{https://github.com/twitter-research/tgn}{\textbf{750 Github stars}}. While at Twitter, I successfully developed multiple ML models for the "\textbf{who-to-follow}" recommendation feature, resulting in an increase of over \textbf{1.2M follow actions daily}.
    % \end{onecolentry}

    \section{Education}
    
    \begin{twocolentry}{
    \textit{Oct 2020 – Feb 2024}}
        \textbf{Imperial College London}

        \textit{Ph.D. in Computer Science} -- \textit{Supervisor: Prof. Michael Bronstein}
    \end{twocolentry}

    \vspace{0.10 cm}
    \begin{onecolentry}
        \begin{highlights}
            \item My research focused on the development of new \textbf{Graph Neural Networks} methods for large-scale real-world applications
            \item My work has been published at top conferences including ICML, ICLR, AAAI and RecSys
        \end{highlights}
    \end{onecolentry}

    \vspace{0.2 cm}

    \begin{twocolentry}{
    \textit{Oct 2018 – Jun 2019}}
        \textbf{University of Cambridge}

        \textit{Master in Advanced Computer Science} -- \textit{Supervisor: Prof. Pietro Liò}
    \end{twocolentry}

    \vspace{0.10 cm}
    \begin{onecolentry}
        \begin{highlights}
            \item Graduated with \textbf{Distinction} (\textbf{4.0 GPA})
            \item Thesis: "\textbf{Graph Deep Learning for ncRNA data}" (\href{https://arxiv.org/abs/1905.06515}{published} at KDD 2019 GDL workshop)
        \end{highlights}
    \end{onecolentry}

    \vspace{0.2 cm}

    \begin{twocolentry}{
    \textit{Oct 2015 – Jun 2018}}
        \textbf{Imperial College London}

        \textit{Bachelor in Computer Science}
    \end{twocolentry}

    \vspace{0.10 cm}
    \begin{onecolentry}
        \begin{highlights}
            \item Graduated with \textbf{First Class Honors} (\textbf{4.0 GPA})
        \end{highlights}
    \end{onecolentry}

    \section{Experience}

    \begin{twocolentry}{
    \textit{NYC (remote)}
        
    \textit{Jan 2024 – Present}}
        \textbf{Machine Learning Researcher}
        
        \textit{VantAI}
    \end{twocolentry}

    \vspace{0.10 cm}
    \begin{onecolentry}
        \begin{highlights}
            \item \textbf{Co-led development of \href{https://www.vant.ai/neo-1}{Neo-1}}, a state-of-the-art all-atom latent diffusion model for multimodal structure prediction and de-novo generation of biomolecular complexes, trained on 100+ GPUs.
            \begin{itemize}
                \item \textbf{Model Design}: Led the design and exploration of model architectures, diffusion formulations, and inference methods, establishing efficient experimentation workflows.
                \item \textbf{Data Curation}: Contributed to the development of high-quality, leakage-free biomolecular datasets (see \href{https://www.plinder.sh/}{Plinder}, \href{https://www.pinder.sh/}{Pinder}) used to train the model.
                \item \textbf{Mentoring}: Mentored and supervised 3 interns who made substantial contributions to the project.
                \item \textbf{Cross-functional Collaboration}: Collaborated closely with computational biologists and chemists to assess and enhance model applicability and accuracy.
            \end{itemize}
        \end{highlights}
    \end{onecolentry}

    \vspace{0.2 cm}

    \begin{twocolentry}{
    \textit{London}
        
    \textit{Jun 2019 – Feb 2023}}
        \textbf{Machine Learning Researcher}
        
        \textit{Twitter}
    \end{twocolentry}

    \vspace{0.10 cm}
    \begin{onecolentry}
        \begin{highlights}
            \item Research and development of \textbf{graph neural networks}, working on both \textbf{fundamental research} and \textbf{production systems}
            \item Successfully \textbf{scaled GNN architectures} to process Twitter's social graph with \textbf{tens of billions of edges}
            \item Designed and deployed an \textbf{embedding-based recommendation system} for the "who to follow" feature, resulting in \textbf{1.2M+ additional daily follow actions} on the platform
        \end{highlights}
    \end{onecolentry}

    \vspace{0.2 cm}

    \begin{twocolentry}{
    \textit{London}
        
    \textit{Mar 2019 – Jun 2019}}
        \textbf{Machine Learning Researcher}
        
        \textit{Fabula AI}
    \end{twocolentry}

    \vspace{0.10 cm}
    \begin{onecolentry}
        \begin{highlights}
            \item Developed \textbf{novel graph deep learning models} for \textbf{fake news classification}
            \item Fabula AI was acquired by Twitter in June 2019
        \end{highlights}
    \end{onecolentry}

    \vspace{0.2 cm}

    \begin{twocolentry}{
    \textit{California}
        
    \textit{Jul 2017 – Oct 2017}}
        \textbf{Software Engineering Intern}
        
        \textit{Google}
    \end{twocolentry}

    \vspace{0.10 cm}
    \begin{onecolentry}
        \begin{highlights}
            \item Worked in the \textbf{Google Play Store} infrastructure team
        \end{highlights}
    \end{onecolentry}

    \section{Selected Publications}

    \begin{onecolentry}
        \begin{highlightsforbulletentries}
            \item \textit{J. Durairaj et al.} \href{https://arxiv.org/abs/2407.12269}{PLINDER: The Protein-Ligand Interactions Dataset and Evaluation Resource}. \textit{ICML ML for Life and Material Science Workshop 2024}. Comprehensive dataset and benchmark for protein-ligand interaction prediction.

            \item \textit{E. Rossi et al.} \href{https://arxiv.org/abs/2305.10498}{Edge Directionality Improves Learning on Heterophilic Graphs}. \textit{Learning on Graphs Conference 2023}. Novel framework for graph neural networks on directed graphs, showing significant improvements on heterophilic graphs. 

            \item \textit{B. Chamberlain, S. Shirobokov, E. Rossi et al.} \href{https://arxiv.org/abs/2209.15486}{Graph Neural Networks for Link Prediction with Subgraph Sketching}. \textit{ICLR 2022, Oral (top 5\%)}. First GNN model to scale link prediction to graphs with millions of nodes thanks to sketching, a hashing technique which enables efficient computation of subgraph statistics.

            \item \textit{E. Rossi et al.} \href{https://arxiv.org/abs/2006.10637}{Temporal Graph Networks for Deep Learning on Dynamic Graphs}. \textit{ICML 2020 GRL Workshop}. Novel model for deep learning on dynamic graphs ($>$1k \href{https://github.com/twitter-research/tgn}{Github stars}). Adopted by \href{https://memgraph.com/blog/amazon-user-item-recommender-with-tgn-and-memgraph}{memgraph} (graph analytics company) and used by GraphCore to \href{https://www.graphcore.ai/posts/accelerating-and-scaling-temporal-graph-networks-on-the-graphcore-ipu}{benchmark their hardware}.

            \item \textit{E. Rossi$^*$, F. Frasca$^*$ et al.} \href{https://arxiv.org/abs/2004.11198}{SIGN: Scalable Inception Graph Neural Networks}. \textit{ICML 2020 GRL Workshop}. First graph deep learning model to scale to graphs with billions of edges. Used in \href{https://medium.com/airbnb-engineering/graph-machine-learning-at-airbnb-f868d65f36ee}{production by Airbnb}.
        \end{highlightsforbulletentries}
    \end{onecolentry}

    \section{Projects and Awards}

    \begin{twocolentry}{
        
        
    \textit{Jun 2018 – Mar 2023}}
        \textbf{LeadTheFuture}
        
        \textit{Co-Founder}
    \end{twocolentry}

    \vspace{0.10 cm}
    \begin{onecolentry}
        \begin{highlights}
            \item \href{http://leadthefuture.tech/}{\textbf{LeadTheFuture}} is a \textbf{non-profit mentoring organization} and the \textbf{largest merit-based STEM community in Italy}, with more than \textbf{200 mentors} (engineers, researchers, and entrepreneurs) and \textbf{500 selected mentees}
            \item Our work has been featured in \href{https://forbes.it/2020/03/01/stage-da-google-o-alla-nasa-con-leadthefuture-idea-di-tre-ragazzi-italiani/}{Forbes}
        \end{highlights}
    \end{onecolentry}

    \vspace{0.2 cm}

    \begin{twocolentry}{
        
        
    \textit{}}
        \textbf{Talks and Blog Posts}
    \end{twocolentry}

    \vspace{0.10 cm}
    \begin{onecolentry}
        \begin{highlights}
            \item I have written a series of blog posts on my research (\href{https://emanuelerossi.co.uk/blog/}{full list}) and have been invited to give over 10 talks (\href{https://emanuelerossi.co.uk/presentations/}{full list})
        \end{highlights}
    \end{onecolentry}

\end{document}